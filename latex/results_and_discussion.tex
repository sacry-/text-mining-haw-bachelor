\epigraph{\emph{
  ``Simple models and a lot of data trump more elaborate models based on less data.''
}}{ Peter Norvig }

In this section the results of this thesis are discussed. The implementations are compared and evaluated. What is problematic and what worked out well? What can we conclude by now?

\section{Evaluation Analysis}
\section{Pros and Cons}
\section{Conclusion}
  Some final words about measurement. It is highly debatable if any of the semantic enhancements has a measureable impact. If a user however views a website with summarizations of the daily news it makes a difference if phrases and references can be made to Wikipedia. It makes a difference in lexical diversity. Topic browsers with external knowledge sources are more diverse than browsers without. This cannot be measured in a mathematical way. The impact might be very small. The reason, the impact is so low is rather obvious: Without knowledge sources and Wordnet ontologies most clustering algorithms work just fine. Enhancing a good running model with additional feature selection heuristics can only make it better if there is mathematical proof that it does. It can certainly not proof if a human likes or dislikes the outcome.


