\documentclass[
  draft=true,
  paper=a4,
  twoside=false,
  fontsize=11pt,
  headsepline,
  BCOR10mm,
  DIV11
]{scrbook}

\usepackage[T1]{fontenc}
\usepackage[utf8]{inputenc}
\usepackage[ngerman, english]{babel}
\usepackage{libertine}
\usepackage{pifont}
\usepackage{microtype}
\usepackage{textcomp}
\usepackage[german,refpage]{nomencl}
\usepackage{setspace}
\usepackage{makeidx}
\usepackage{listings}
\usepackage{natbib}
\usepackage[ngerman,colorlinks=true]{hyperref}
\usepackage{soul}
\usepackage{hawstyle}
\usepackage{lipsum}

\colorlet{BackgroundColor}{gray!20}
\colorlet{KeywordColor}{blue}
\colorlet{CommentColor}{black!60}
\colorlet{HeadColor}{gray!60}
\colorlet{Color1}{blue!10}
\colorlet{Color2}{white}

\HAWifprinter{
  \colorlet{BackgroundColor}{gray!20}
  \colorlet{KeywordColor}{black}
  \colorlet{CommentColor}{gray}
  \colorlet{HeadColor}{gray!60}
  \colorlet{Color1}{gray!40}
  \colorlet{Color2}{white}
}{}

\lstset{
  numbers=left,
  numberstyle=\tiny,
  stepnumber=1,
  numbersep=5pt,
  basicstyle=\ttfamily\small,
  keywordstyle=\color{KeywordColor}\bfseries,
  identifierstyle=\color{black},
  commentstyle=\color{CommentColor},
  backgroundcolor=\color{BackgroundColor},
  captionpos=b,
  fontadjust=true
}

\lstset{escapeinside={(*@}{@*)}, % used to enter latex code inside listings
  morekeywords={uint32_t, int32_t}
}

\ifpdfoutput{
  \hypersetup{bookmarksopen=false,bookmarksnumbered,linktocpage}
}{}

\DeclareRobustCommand{\cxx}{C\raisebox{0.25ex}{{\scriptsize +\kern-0.25ex +}}}

\clubpenalty=10000
\widowpenalty=10000
\displaywidowpenalty=10000

\hyphenation{}

\typearea[current]{last}

\makeindex
\makenomenclature

\begin{document}
  \HAWThesisProperties{
    Author={Matthias Nitsche},
    Title={Continuous Clustering for a Daily News Summarization System},
    EnglishTitle={Continuous Clustering for a Daily News Summarization System},
    ThesisType={Bachelorarbeit},
    ExaminationType={Bachelorprüfung},
    DegreeProgramme={Bachelor of Science Angewandte Informatik},
    ThesisExperts={Prof. Dr. Michael Neitzke \and Prof. Dr. Olaf Zukunft},
    ReleaseDate={02. February 2016}
  }


  \frontmatter
    \maketitle
    \onehalfspacing

    \HAWAbstractPage
  {
    Clustering, Cluster Analyse, 
    Dokument Clustering, Vector Space Model,
    Partitionelles Clustering,
    Hierarchisches Clustering,
    Probabilistic Topic Modeling, 
    Text Zusammenfassung,
    Text Mining, Data Mining,
    Machinelles Lernen, Unüberwachtes Lernen, 
    Information Retrieval
  }
  { 
    Für eine Maschine ist es schwer, ohne die Supervision eines menschlichen Expertens text zu interpretieren. Techniken des Text Minings und clustern als Ansatz von unsupervisierten Lernen, um Text aus Zeitungen in Kategorien und zu Ereignissen in der realen Welt zu gruppieren, ist zentral in dieser Arbeit. Zusätzlich, wurde ein funktionierendes Datenverarbeitungssystem, zum herunterladen und verarbeiten von Zeitungsartikeln, entwickelt, um clustering Algorithmen eine Grundlage zu geben. In kurz, die präsentierten Selektionsstrategien und clustering Algorithmen haben eine ähnliche Wirkung.
  }
  { 
    Clustering, Cluster Analysis, 
    Document Clustering, Vector Space Model,
    Partitional Clustering,
    Hierarchical Clustering,
    Probabilistic Topic Modeling,
    Summarization,
    Text Mining, Data Mining,
    Machine Learning, Unsupervised Learning, 
    Information Retrieval
  }
  {
    Interpreting and summarizing textual content without the supervision of human experts is subject of this thesis. Using techniques of text mining and document clustering as an approach of unsupervised machine learning, grouping textual content of online newspaper articles, into coherent categories and real world events is subject of this thesis. Additionally, building a functioning data pipeline for scraping and preprocessing newspaper articles, feeding clustering algorithms, shows promising results. In short, the presented feature selection and clustering strategies yield similar effects.
  }




    \newpage

    \singlespacing
    \tableofcontents

    \newpage
    \listoftables
    \listoffigures
    %\lstlistoflistings


  \mainmatter
    \onehalfspacing
    \typeout{===== File: chapter 1}

    \chapter{Introduction}
      \label{chapter:introduction}
      
\cite{NewsBlaster2002}
\cite{ColumbiaExperimentsSum2002}
\cite{ColumbiaMultiDoc2001}
\cite{ColumbiaMultiSumWang2011}
\cite{Simfinder2001}

\cite{SumLSASteinberger2004}
\cite{SumLSA2001}

\cite{blei2011introduction}
\cite{SemanticTopicModels2011}
\cite{TopicModelsBlei2012}
\cite{ProbTopicModelsSteyvers2006}
\cite{PLSA2001}
\cite{LDA2003}
\cite{NMF1999}

\cite{NLPBookJurafsky2000}
\cite{ClusteringBooAggarwalk2013}
\cite{ClusterAlgoSurveyIBM}
\cite{IRBook2008}

\cite{ClusterRefinementModelSelect@2002@Liu}
\cite{FeatureSelection}

\cite{Strang2009}

\cite{ScikitLearn}
\cite{gensim2010}

\cite{DeerwesterLSI1990}
\cite{PCA2010}

\cite{NextFrontierClustering2013}
\cite{NounPhraseSemanticClustering@2009@Zheng}
\cite{WordNetAndFuzzyAssociation@2010@Chen}

      \ipsum

    \chapter{State of the Art}
      \label{chapter:state_of_the_arts}
      \lipsum

    \chapter{Data pipeline}
      \label{chapter:data_pipeline}
      \lipsum

    \chapter{Preprocessing}
      \label{chapter:preprocessing}
      \lipsum

    \chapter{Feature Selection}
      \label{chapter:feature_selection}
      \lipsum

    \chapter{Clustering algorithms}
      \label{chapter:clustering_algorithms}
      \lipsum

    \chapter{Implementation}
      \label{chapter:implementation}
      \lipsum

    \chapter{Results and Discussion}
      \label{chapter:results}
      \lipsum

    \chapter{Outlook}
      \label{chapter:outlook}
      \lipsum

  \backmatter
    \typeout{===== Section: literature}
    \bibliographystyle{apalike}
    \bibliography{thesis}

    \addcontentsline{toc}{chapter}{Glossary}
    \renewcommand{\nomname}{Glossary}
    \markboth{\nomname}{\nomname}
    \printnomenclature
    \clearpage

    \typeout{===== Section: index}
    \printindex

    \HAWasurency

\end{document}
