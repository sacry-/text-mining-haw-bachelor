\documentclass[
  draft=true,
  paper=a4,
  twoside=false,
  fontsize=11pt,
  headsepline,
  BCOR10mm,
  DIV11
]{scrbook}

\usepackage[T1]{fontenc}
\usepackage[utf8]{inputenc}
\usepackage[ngerman, english]{babel}
\usepackage{libertine}
\usepackage{pifont}
\usepackage{microtype}
\usepackage{textcomp}
\usepackage[german,refpage]{nomencl}
\usepackage{setspace}
\usepackage{makeidx}
\usepackage{listings}
\usepackage{natbib}
\usepackage[ngerman,colorlinks=true]{hyperref}
\usepackage{soul}
\usepackage{hawstyle}
\usepackage{lipsum}

\colorlet{BackgroundColor}{gray!20}
\colorlet{KeywordColor}{blue}
\colorlet{CommentColor}{black!60}
\colorlet{HeadColor}{gray!60}
\colorlet{Color1}{blue!10}
\colorlet{Color2}{white}

\HAWifprinter{
  \colorlet{BackgroundColor}{gray!20}
  \colorlet{KeywordColor}{black}
  \colorlet{CommentColor}{gray}
  \colorlet{HeadColor}{gray!60}
  \colorlet{Color1}{gray!40}
  \colorlet{Color2}{white}
}{}

\ifpdfoutput{
  \hypersetup{bookmarksopen=false,bookmarksnumbered,linktocpage}
}{}

\DeclareRobustCommand{\cxx}{C\raisebox{0.25ex}{{\scriptsize +\kern-0.25ex +}}}

\clubpenalty=10000
\widowpenalty=10000
\displaywidowpenalty=10000

\hyphenation{}

\typearea[current]{last}

\makeindex
\makenomenclature

\begin{document}
  \HAWThesisProperties{
    Author={Matthias Nitsche},
    Title={Continuous Clustering for a Daily News Summarization System},
    EnglishTitle={Continuous Clustering for a Daily News Summarization System},
    ThesisType={Bachelorarbeit},
    ExaminationType={Bachelorprüfung},
    DegreeProgramme={Bachelor of Science Angewandte Informatik},
    ThesisExperts={Prof. Dr. Michael Neitzke \and Prof. Dr. Olaf Zukunft},
    ReleaseDate={02. February 2016}
  }


  \frontmatter
    \maketitle
    \onehalfspacing

    \HAWAbstractPage
  {Clustering, Text Mining, Natürliche Sprachverwarbeitung, Data Engineering, Dokument Klassifikation, Computer Linguistik, Python, NLTK}
  {Dieses Dokument \ldots}
  {Clustering, Text Mining, Natural Language Processing, Data Engineering, Document Classification, Computational Linguistics, Python, NLTK}
  {This document \ldots}
    \newpage

    \singlespacing
    \tableofcontents

    \newpage
    \listoftables
    \listoffigures
    %\lstlistoflistings


  \mainmatter
    \onehalfspacing
    \typeout{===== File: chapter 1}

    \chapter{Introduction}
      \label{chapter:introduction}
      
\cite{ClusterRefinementModelSelect}
\cite{ClusterAlgoSurveyIBM}
\cite{Jurafsky2000nlp}
\cite{NounPhraseSemanticClustering}
\cite{ClusterRefinementModelSelect}
\cite{DocClusterMultiSum}
\cite{WordNetAndFuzzyAssociation}
      \ipsum

    \chapter{State of the Art}
      \label{chapter:state_of_the_arts}
      \lipsum

    \chapter{Data pipeline}
      \label{chapter:data_pipeline}
      \lipsum

    \chapter{Preprocessing}
      \label{chapter:preprocessing}
      \lipsum

    \chapter{Feature Selection}
      \label{chapter:feature_selection}
      \lipsum

    \chapter{Clustering algorithms}
      \label{chapter:clustering_algorithms}
      \lipsum

    \chapter{Implementation}
      \label{chapter:implementation}
      \lipsum

    \chapter{Results and Discussion}
      \label{chapter:results}
      \lipsum

    \chapter{Outlook}
      \label{chapter:outlook}
      \lipsum

  \backmatter
    \typeout{===== Section: literature}
    \bibliographystyle{apalike}
    \bibliography{thesis}

    \addcontentsline{toc}{chapter}{Glossary}
    \renewcommand{\nomname}{Glossary}
    \markboth{\nomname}{\nomname}
    \printnomenclature
    \clearpage

    \typeout{===== Section: index}
    \printindex

    \HAWasurency

\end{document}
