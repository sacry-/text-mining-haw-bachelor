\epigraph{\emph{
  ``Where there is matter, there is geometry.''
}}{ Johannes Kepler }

Text mining and natural language processing have long been studied in the field of artificial intelligence. Within the last decade models that were computationally expensive were reopened for discussion. Machine learning aids in this task by approximating models for structured and unstructured data. Recognizing patterns, retrieving information or classifying content of interest for huge datasets, often scaled over thousands of machines. This process is called knowledge discovery. By continuously clustering daily news articles to summarize them the task at hand is simple. Reduce the noise surrounding textutal documents by presenting the useful fraction of information. Connect documents in their topical logics to enhance the understanding and reasoning about them. Mining documents, preprocessing them into suitable representations and grouping their form to detect underlying patterns that connect documents is the primary goal of this thesis.\\
Newspaper articles however need to be produced by humans. While automation and easy retrieval can be of utmost importance, news are created by single human beings and authors. The whole work of text mining in that sense is to scale suitable algorithms to datasets that are too enormous to be comprehended by a single person. In the end, without authors there is no text, therefore the highest good at hand are the documents by the authors.\\
In the following we will examine the field of unsupervised learning with respect to clustering examplified by a news clustering system. The objective is to build parts of an automatic summarization system that scrapes newspapers, preprocesses the content and groups them into clusters for summarization purposes.\\
Also in the name of Johannes Keppler ``Where there is matter, there is geometry.'', lining up perfectly that documents and words have matter and geometry is there for the rescue. More on this, later.

\section{Machine Learning}
  Artifical intelligence is the field of study asking the question: Are computers capable of intelligent behaviour? By intelligent it is refered to how an agent can be programmed to react and act to an environment by maximizing the chances of success in a particular task. Machine learning is a subfield of artifical intelligence.\\
  A computer program is said to learn in the context of performing a task if its performance with respect to some measure improves with experience. In the context of this thesis machine learning is closely related to pattern recognition - the act of teaching a program to react to or recognize patterns. It can be split into three broader categories namely supervised learning, unsupervised learning and reinforcement learning. Supervised learning is the machine learning task of inferring a function from labeled training data. Those labels are typically sorted into classes defined by expert human knowledge. Unsupervised learning on the opposite is the task of finding patterns in unstructured data. That means, there is no prior knowledge involved and the algorithms approximate solutions that show underlying patterns and connections. Reinforcement learning is learning what to do - how to map situations to actions -so as to maximize a numerical reward signal.\\
  In this thesis we are mainly confronted with unsupervised learning. Unsupervised learning is special in a sense that we do not know what we want to find. Finding patterns can mean anything and most often only human beings are capable of interpreting the quality of a result. As a result most algorithms found in unsupervised learning are NP-Hard problems that approximate solutions given some objective maximizing or minimizing function.

\section{Structure of the thesis}
  In chapter 2 the basics are examined. Theoretical foundations that are common in this area are explained and assumed to be understood.\\
  In chapter 3 there will be an overview of the ``News-Clusty'' data pipeline in light of the Columbia Newsblaster system.\\
  In chapter 4 it is explained in detail how features for clustering algorithms are selected and enhanced.\\
  Chapter 5 introduces experiments and evaluation of different clustering strategies in light of the ``News-Clusty'' system.\\
  Chapter 6 we will look at the results and discuss problems and chances.\\
  The final chapter 7 is there for an outlook. What is missing? What can be done. Furhter reading and everything related to future work.

