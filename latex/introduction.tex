\epigraph{\emph{
  ``Where there is matter, there is geometry.''
}}{ Johannes Kepler }

\paragraph{} Text mining and natural language processing have long been studied in the field of artificial intelligence. Within the last decade, models that are computationally expensive, were reopened for discussion. Machine learning aids in this task by approximating models for structured and unstructured data. Recognizing patterns, retrieving information or classifying content of interest for huge data sets, often scaled over thousands of machines. This process is called knowledge discovery. By continuously clustering daily news articles to summarize them, the task at hand is simple: Reducing noise which surrounds textual documents by presenting the useful fraction of information. The useful fraction of information can be anything from topic detection to summarization providing additional reasoning for humans. Mining documents, preprocessing them into suitable representations and grouping their form to detect underlying patterns that connect documents is the primary goal of this thesis.

\paragraph{} It goes without saying that newspaper articles are created by humans. While automation and easy retrieval can be of utmost importance, news are created by authors, journalists, generally professional writers. The main purpose of text mining is to scale suitable algorithms to data sets, that are too enormous to be comprehended by a single person. Without authors there is no text, therefore the highest good at hand are the documents by the authors.

\paragraph{} In the following we will examine the field of unsupervised learning with respect to clustering exemplified by a news clustering system. The objective is to build parts of an automatic summarization system that scrapes newspapers, preprocesses the content and groups them into clusters for summarization purposes.\\
We will soon see that assumptions from linear algebra and geometry will be of utmost importance to understand text representation. Essentially it is all about smart counting. This sentiment lines up with the quote by Johannes Keppler ``Where there is matter, there is geometry.''

\section{Machine Learning}
\label{sec:ml_intro}
  \paragraph{} Artificial intelligence is the field of study, asking the question: Are computers capable of intelligent behavior? By intelligent it is referred to how an agent can be programmed to react and act to an environment by maximizing the chances of success in a particular task. \emph{Machine learning} is a sub field of artificial intelligence. The distinction mainly lies in the training of models that learn a good representation of data given a hypothesis.

  \paragraph{} A computer program is said to learn in the context of performing a task if its performance with respect to some measure improves with experience. In the context of this thesis, machine learning is closely related to \emph{pattern recognition} - the act of teaching a program to react to or recognize patterns. It can be split into three broader categories namely supervised learning, unsupervised learning and reinforcement learning. \emph{Supervised learning} is the machine learning task of inferring a function from labeled training data. Those labels are typically sorted into classes defined by expert human knowledge. \emph{Unsupervised learning} is the task of finding patterns in unstructured data. That means, there is no prior knowledge involved and the algorithms approximate solutions that show underlying patterns and connections. \emph{Reinforcement learning} is learning what to do - how to map situations to actions - so as to maximize a numerical reward signal.

  \paragraph{} In this thesis we are mainly confronted with unsupervised learning. Unsupervised learning is special, in the sense that we do not know what we want to find. Finding patterns can mean anything and most often only human beings are capable of interpreting the quality of a result. As such, unsupervised learning focuses on algorithms that approximate optimal solutions to \emph{NP-Hard} problems by an objective maximizing or minimizing cost function.

\section{Structure of the thesis}
  In \emph{chapter 2}, theoretical foundations and basics are examined. Helpful but not needed is prior knowledge in linear algebra, statistics and algorithms.\\
  In \emph{chapter 3}, we introduce the news clustering system ``News-Clusty'', a data pipeline, in comparison to the \emph{Columbia Newsblaster system}.\\
  In \emph{chapter 4}, feature selection strategies for clustering algorithms are formally described.\\
  In \emph{chapter 5}, experiments and evaluation of different clustering routines are presented.\\
  In \emph{chapter 6}, we will discuss results, problems and chances.\\
  The final \emph{chapter 7} sums up the thesis, gives future directions and provides additional material for reading.

