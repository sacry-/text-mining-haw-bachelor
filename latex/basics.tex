\epigraph{\emph{
  ``If I have seen further it is by standing on the shoulders of giants.''
}}{ Isaac Newton }

The goal of this section is to give some intuition and the necessary theoretical background for the following chapters. The areas where clustering problems arise are huge. It provides solutions to problems like market segmentation, classification, document organization or indexing.\\
Firstly we will have a look at the definition of clustering and summarization. How they are related and the variety of possibilities this imposes.
Secondly the vector space model (VSM) is introduced. It contains all information about how to represent documents in a vectorized form. Of special interest are enhanced models which reduce the dimensionality of documents by singular value decomposition (SVD).
Thirdly traditional clustering algorithms from the hierarchical (Ward, Birch) and partitional (K-Means, Mean-Shift) family will be presented.
Closely related are the generative models. These methods can be used as a kind of clustering algorithm and are highly useful in several steps of traditional clustering. They can be used as dimensionality reduction techniques as well.
Lastly some quality measures of clusters based on internal measures (without ground truth labels) and external measures (with explicit labelling of the ground truth) are explained.


\section{Clustering and Summarization}
  
  \textbf{Clustering} as defined by \cite{ClusterAlgoSurveyIBM} is finding groups of similar objects in the data with a defined similarity function between objects. The granularity of the features can vary:

  \begin{itemize}
    \item \emph{Sentence based} - A document d is split into sentences so clustering reveals the most coherent groups of sentences that are closely related.
    \item \emph{Collection of documents} - A collection of documents d (corpora) is grouped to get groups of documents that are closely related
    \item \emph{Stream of documents} - The same as clustering copora with the constraint that over time the size of documents grow.
  \end{itemize}

  The following is an illustration of a typical cluster result. The larger points correspond to cluster centers while the smaller points correspond to individual documents adhering to the cluster centers colors. 

  \begin{figure}[h!]
    \centering
      \includegraphics[width=0.7\textwidth]{clustering_intro.png}
      \caption{"K-means visualisation with applied dimensionality reduction"}
      \label{clustering_intro}
  \end{figure}

  Document clustering on large corpora can be seen as a summarization of the underlying concepts. The representation of documents as feature vectors is described with the vector space model in the next section.\\
  \textbf{Automatic text summarization} on the other hand, is the process of reducing textual content to the most important concepts in a readable, formatted form to the user \cite{SumEvaluation2001}.\\
  This results in a few possibilities where clustering works great as a preprocessing step for summarization.
  
  \begin{description}
    \item[First] Clustering groups that have a \emph{higher density of information} resulting in a grouped input for summarizers.
    \item[Second] Grouping the \emph{latent topics} accross and within documents to create a meta concept of closely related documents
    \item[Third] Classify documents into \emph{categories} in a semi supervised way to construct hierarchies of relationships
    \item[Fourth] Finding \emph{outliers} that will not highly contribute to the summarization
  \end{description}

  Aside clustering itself can be seen as a summarization as well. Clustering can lead to well formed topical browsers where users can interact with a graphical user interface to browse topics in a more coherent and semantic way see \cite{Carrot2Search2003}.

  \paragraph{Supervision}

    As opposed to unsupervised learning strategies such as clustering, supervised learning classifies some input based on a provided ground truth. That is for an input \emph{x} there are labels \emph{y} that describe the class they are in. Supervision can be done by explicitly classifying the documents before the clustering. The input is then split into \emph{n} classes. Then each class can be individually clustered. Often however this is no option. We need to manually label all documents. This can be time consuming and error prone. Often several labellers are needed to crossvalidate human bias.
    With this in mind there are two options on how to label unseen or new data:

      \begin{itemize}
        \item Use a supervised classification algorithm to automatically label unlabelled data. A prerequisite is to have a labelled training set and to have a lot of data. To name a few candidates: \emph{Multinominal/Gaussian) Naive Bayes (NB)}, \emph{Multivariate Logistic/Linear Regression}, \emph{Neural Networks (ANN)}, \emph{Support Vector Machines (SVM)} or \emph{Random Forests (RF)}.
        \item Use an unsupervised clustering algorithm to automatically label unlabelled data. This can be done by first forming clusters and then merging the nearest clusters until k distinct categories remain. Usually the merging criterion can be controlled by some threshold and high variance documents are sorted out into an outlier cluster.
      \end{itemize}

    Usually by clustering we mean automatic detection of the grund truths. Often this is to shallow and does not lead to labels with a high confidence. In the domain of document clustering all information that provide some context are critical and should be used.

\newpage{}
\section{Vector Space Model (VSM)}
  
  \paragraph{}
    The vector space model is directly derived from the vector space subject to linear algebra. If we talk about vector space we often refer to the euclidean vector space where the dimensions are typically up to 3 dimensions. All dimensions higher than 3 are hard to imagine. From the view of linear algebra the vector space consists of linear combinations that are solved by $Ax = b$ by some matrix decomposition step. In the context of document clustering the vector spaces typically far exceed 3 dimensions up to $\mathbf{R}^{n}$. For a proper introduction to linear algebra see \cite{Strang2009}.

  \begin{figure}[h!]
    \centering
      \includegraphics[width=0.7\textwidth]{vsm.png}
      \caption{"Vector space model"}
      \label{vsm_pic}
  \end{figure}

  \paragraph{}
    The vector space model in the text domain has the meaning that each word is a component of a document. If a document has 100 distinct words, the resulting document vector is in 100th dimensional space. If a second document has 50 distinct words, independant of the first document, both vectors are now in 150th dimensional space. That is every new word will be concatenated to the existing document sets. This is the ``bag of words'' model which is detailed in the next section.

  \subsection{Notation}
    Before moving on we have to denote some notations and definitions. 
    A corpus \emph{C} is defined as a collection of \emph{m} documents

    \begin{equation}
      C = \{d_1, d_2 \: .. \: d_m\}
    \end{equation}

    A document \emph{d} contains words \emph{w} such that
    
    \begin{equation}
      d = \{w_1, w_2 \: .. \: w_i\}
    \end{equation}
    
    Where \emph{i} is \emph{|d| (length of document d)}. Note that each document is not a text but a sequence of words. That is the result of preprocessing where texts are tokenized to individual word tokens. Often special characters are filtered out as well.\\
    A dictionary \emph{D} contains all the distinct words from each document.

    \begin{equation}
      D = \{w_1, w_2 \: .. \: w_n\}
    \end{equation}

  \subsection{Bag of words}
    The bag of words assumption says that a corpus can be represented as a count or word occurence matrix. That means \emph{m} documents form a subspace in an \emph{m x n} matrix where \emph{m} denotes the corpus size and \emph{n} the dictionary of the words. Typically the assumption is binary, that is an occurence of word \emph{i} in a document \emph{j} is set to 1 else set to 0.

    \begin{table}[h!]
      \centering
      \begin{tabular}{c|c|c|c}
        \multicolumn{1}{r|}{} & \multicolumn{3}{c}{Words} \\
        \cline{1-4}
        Documents &   politics &   corruption &  policy  \\
        \hline
        document1 &    1 (2)   &     1 (1)    &   0 (0)  \\
        document2 &    1 (4)   &     0 (0)    &   1 (2)  \\
        document3 &    1 (1)   &     1 (6)    &   0 (0)  \\
      \end{tabular}\\
      \caption{"Document term matrix, the numbers in brackets is the actual count"}
    \end{table}

    Normally we have a lower \emph{m} and a much higher \emph{n} resulting in highly sparse vectors with a lot of zeros, typically 99\%. We can also conclude that the order of words is not kept. That means highly correlating words like \emph{New} and \emph{York} are not accounted for. In this case they can actually refer to the verb \emph{new} and \emph{York} as a city in \emph{Great Britain}. A bigram e.g. \emph{(New, York)} would capture the concept \emph{New York}. Bigrams, trigrams or generally ngrams are not taken into account. Ngrams transform a sequence of words by \emph{n} such that 

      \begin{equation}
        f(n = 2, \{w_1, w_2, w_3\} \in d) \to \{(w_1, w_2),(w_2,w_3)\}
      \end{equation}

    Adding ngrams to a to a document term matrix can greatly enhance similiarity between two documents. With single words and ngrams the memory requirement is \emph{n+(n-1)}. If the data is already extremely sparse this will not help much but can increase the semantic effect on documents that share similar word combinations. There are other much more complex models e.g. \emph{noun phrases} or \emph{named entities} that can grasp this intuition as well.

    The document term matrix can be enhanced by taking the count of the occuring words instead of just labelling it by occurence. This is called the raw frequency and can be normalized in a couple of ways by the term frequency (tf) model
     
      \begin{equation}
        f(t,d) = c
      \end{equation}

    where \emph{c} is the total count of term \emph{t} occuring in document \emph{d}.
    And a general term frequency function that helps against long document bias by normalizing with the maximum frequency of any occuring word in d.

    \begin{equation}
      tf(t,d) = 0.5 + \frac{0.5 * f(t,d)}{max(f(w,d)\, \forall w \in d)}
    \end{equation}

    The term frequency model can be further advanced by caculating the inverse document frequency (idf).

    \begin{equation}
      idf(t, C) = log(\frac{|C|}{|\forall d \in C : t \in d|})
    \end{equation}

    Where |C| is the size of the corpus and we check for every document in the corpus if the term occurs in the document. The intuition is: How often does a term occur in other documents. If a term appears more often then it is a common word such as ``the'', whereas ``super-symmetry'' might be a rare word. It is therefore some measure of importance. Because idf and tf only measure either importance accross all documents or importance of one document, we need to find words that are not rare and not common either. Generalizing the term-frequency and inverse-document-frequency we obtain the tf-idf:

    \begin{equation}
      tfidf(t, d, D) = tf(t, d) * idf(t, D)
    \end{equation}

    The tf-idf has a high score, if a term occurs often in a single document and less often in other documents. This translates to the notion that a term represents the current document better than other terms and are therefore highly discriminative words.\\

    There are other models such as graph based or tree based approaches. For the porpuse of this thesis these models are left out. @TODO:Ref

    One of the big problems with the vector space model is that feature inflation or feature explosion arises quickly. If documents have a high variance between other documents and the connections between two documents are small the sparsity can go up to 100\%. That means that no document has overlapping words and each document only accounts for the words that were originally in the document. Think of it another way, suppose one has 3 documents in 3 dimensional space with each axis set to zero but one (1,0,0), (0,1,0), (0,0,1) we have an independant basis. This means that all documents are orthogonal to each other, meeting at no point except at the null point (0,0,0). It is therefore not really possible to derive any connections between those 3 points. Thus each document represents its own topic.
    These kind of problems need a well thought out solution which is presented later in this section.

  \subsection{Similarity and Distances}
    Partitional clustering algorithms commonly work through some objective distance or similarity function between two objects. After lifting the documents into vector space, we are faced with the problem of distance between two documents. Several measures were proprosed mostly stemming from linear algebra so we will briefly review them. Our goal is to have not an \emph{m x n}matrix for \emph{document x term} correlations but an \emph{m x m} matrix for \emph{document x document} similarities. After all the goal is to compare two documents.

    \paragraph{Cosine similarity}
    Given two documents $d_1$ and $d_2$ the classical cosine similarity is defined by

    \begin{equation}
      cos(d_1, d_2) = \frac{d_1 * d_2}{||d_1|| * ||d_2||}
    \end{equation}

    If the documents are already defined unit vectors cosine similarity is just $cos(d_1, d_2) = d_1 * d_2$. The cosine similarity is a measure of orientation, not the magnitude. The angles between documents is compared and thus if the angle is 0° both documents are equal in size and word occurences. On the other hand we often like to measure not only the direction but the magnitude.

    \paragraph{Euclidean distance ($l^2\:norm$)}
    Again given two documents $d_1$ and $d_2$ it is the geometrical distance between two vectors in $\mathbf{R}^n$. This results in the fact that documents running into different directions, e.g. have different angles might as well be very close. It is a geometric measure and accounts for magintude.

    \begin{equation}
      euclid(d_1, d_2) = \sqrt{\sum_{i=1}^{M}(d_1^i * d_2^i)^2}
    \end{equation}

    \paragraph{Manhattan ($l^1\:norm$)}
    Is commonly known under the city block distance measure. Equally squared brackets
    will get you closest to a document by comparing one dimension at a time and summing up their totals.

    \begin{equation}
      manhattan(d_1, d_2) = \sum_{i=1}^{M}|d_1^i - d_2^i|
    \end{equation}

    There are more distance functions such as the Jaccard coefficient that works on intersections of sets and the Chebyshev distance which is a maximizing greedy strategy of the manhattan distance. The underlying concepts of similarity should be clear though. Each coefficient works good on a particular set, however cosine and euclidean distances are the commonly used choices.

  \subsection{Enhancing the Vector Space Model}
    \subsubsection{Singular Value Decomposition (SVD)}
    \subsubsection{Latent Semantic Analysis (LSA)}
    \subsubsection{Principal Component Analysis (PCA)}

\section{Clustering algorithms}

  \subsection{Objective goal}
    EM, Cost functions, general clustering scheme

  \subsection{Hierarchical / Agglomerative clustering}
    \paragraph{Ward, Complete and Average Linkage}
    \paragraph{Birch}

  \subsection{Partitional clustering}
    \paragraph{K-Means}
    \paragraph{Mean-Shift}

    \subsection{Others}
      \begin{enumerate}
        \item \emph{Spectral} - x
        \item \emph{Density} - x
        \item \emph{Grid} - x
      \end{enumerate}


\section{Generative Models}    
  
  \subsection{Topic modelling}
    \paragraph{Bayes Theorem}
    \paragraph{Multinominal Distributions}
    \paragraph{Dirichlet Distributions}
      Chinese Restaurant Process

  \subsection{Methods}
    \subsubsection{Latent Dirichlet Allocation (LDA)}
    \subsubsection{Non Negative Matrix Factorization (NMF)}


\section{Clustering quality measures}

  \subsection{Internal measures}
    Without labels of the ground truth

    \begin{enumerate}
      \item \emph{Silhouette coefficient} - x
      \item \emph{Davies–Bouldin index} - x
      \item \emph{Dunn index} - x
    \end{enumerate}

  \subsection{External measures}
    With labels of the ground truth

    \begin{enumerate}
      \item \emph{F-Measure} - x
      \item \emph{Jaccard index} - x
    \end{enumerate}

